\chapter{Outlook}
Real-time efficient scaling mechanism and automation of processing of analog measurements provide gateway for conducting advanced application stress testing in the stress test environment.
The newly introduced scaling mechanism known as Reverse Polish Notation can be used for further improvement of a scaling function. The gain function introduced by a 
differential operational amplifier itself could be a non-linear function. The differential input/output impedance of an operational amplifier can be of a significant effect 
in calculation of discrete voltage values specfic to control module. The internal circuits of the differential operational amplifier can be analysed and the
contribution from its gain function in performing the scaling can be described using Reverse Polish Notation. The implementation of the Lua interpreter has an influence in the real-time performance of the Reverse Polish Notation because the RPN uses Lua stack to perform the scaling operartion. Lua interpreter even though it utilizes minimum of RAM because of it's implementation in lua, there is a scope to optimise the implementation of the Lua interpreter. Optimisation of the implementaion of Lua interpreter results in the improved execution time performance of the Reverse Polish Notation.    
 