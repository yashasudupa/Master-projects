\chapter{Summary and Outlook}

\section{Summary}

In this sophisticated stress test environment to improve the data processing mechanism, the electrical circuits that are associated with measuring the physical quantities that generate in a semiconductor, are analyzed.
A classical linear scaling mechanism is introduced to the data processing system of a control module by analyzing the measurement circuits.
Also, the Reverse Polish Notation scaling computation function is incorporated into the data processing system of the control module which opens up the possibility of performing a complex level of scaling functions.

The boards that are present in the stress test environment are identified by the extraction of the respective board ids.
The extraction of board ids and their correspondence in the resolution of board information coming from different sources such as test plan and MoPS web server, led to the automation of processing of analog measurements that generate in the semiconductor, irrespective of the stress test application. 

The automation in the processing of analog measurements has simplified the writing procedure of test plans and manual scaling is no more involved within the test plans. 
Human error is minimized because of the automated method of scaling.   
The improved data processing mechanism is implemented and tested in the KAI lab. 

\section{Outlook}

Real-time efficient scaling mechanism and automation of processing of analog measurements provide gateway for conducting advanced application stress testing in the stress test environment.
The newly introduced scaling mechanism known as Reverse Polish Notation can be used for further improvement of a scaling function. The gain function introduced by a 
differential operational amplifier itself could be a non-linear function. The differential input/output impedance of an operational amplifier can be of a significant effect 
in calculation of discrete voltage values specfic to control module. The internal circuits of the differential operational amplifier can be analysed and the
contribution from its gain function in performing the scaling can be described using Reverse Polish Notation. The implementation of the Lua interpreter influences in the real-time performance of the Reverse Polish Notation because the RPN uses Lua stack to perform the scaling operation. Lua interpreter even though it utilizes a minimum of RAM because of its implementation in Lua, there is a scope to optimize the implementation of the Lua interpreter. Optimization of the implementation of the Lua interpreter results in the improved execution time performance of the Reverse Polish Notation. Similarly, the execution time performance of Reverse Polish Notation could significantly be improved by replacing the division operation that takes place during the scaling computation by right shifting 12 times the digital output along with some operation to handle the decimal result. 
